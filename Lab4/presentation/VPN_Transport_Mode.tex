\documentclass[12pt,a4paper]{report}
\usepackage[utf-8]{inputenc}
\usepackage[english]{babel}
\usepackage{graphicx}
\usepackage{float}
\usepackage{geometry}
\usepackage{hyperref}
\usepackage{fancyhdr}
\usepackage{listings}
\usepackage{xcolor}
\usepackage{amsmath}
\usepackage{amssymb}

\geometry{left=2cm,right=2cm,top=2cm,bottom=2cm}

\pagestyle{fancy}
\fancyhf{}
\rhead{VPN Transport Mode}
\lhead{Packet Tracer Activity}
\cfoot{\thepage}

\title{\textbf{Packet Tracer - Configuring VPN Transport Mode}}
\author{}
\date{\today}

\begin{document}

\maketitle

\tableofcontents
\newpage

\chapter{Introduction}

This lab activity demonstrates the critical importance of VPN (Virtual Private Network) encryption in protecting sensitive data transmission. Using Cisco Packet Tracer, we configure a VPN connection and observe the differences between unencrypted and encrypted FTP traffic.

\section{Objectives}

\begin{enumerate}
    \item Observe unencrypted FTP traffic using a packet sniffer
    \item Configure a VPN client on Phil's computer
    \item Verify that FTP traffic is encrypted when transmitted through the VPN tunnel
    \item Understand the security benefits of VPN encryption
\end{enumerate}

\section{Network Topology}

The lab uses a Packet Tracer network with:
\begin{itemize}
    \item \textbf{Phil's Computer:} Client device requiring VPN protection
    \item \textbf{VPN Server:} Gateway for secure tunneling
    \item \textbf{Private FTP Server:} Target server behind VPN
    \item \textbf{Attacker's Computer:} Packet sniffer to capture traffic
\end{itemize}

\chapter{Part 1: Capturing Unencrypted FTP Traffic}

\section{Objective}

Before implementing VPN protection, we first capture unencrypted FTP traffic to show what a cyber criminal could see without VPN encryption.

\section{Initial Configuration Issue}

\begin{figure}[H]
    \centering
    \includegraphics[width=0.85\textwidth]{../screenshots/error\ before\ with\ step\ 1.png}
    \caption{Initial error encountered during setup phase - requiring configuration adjustment}
    \label{fig:initial_error}
\end{figure}

\section{Configuring Phil's Network Settings}

The first step was to configure Phil's computer with proper network settings. Initially, a static IP configuration caused connectivity issues. The solution was to change Phil's network configuration to use DHCP (Dynamic Host Configuration Protocol) for automatic IP assignment.

\begin{figure}[H]
    \centering
    \includegraphics[width=0.85\textwidth]{../screenshots/fixed\ it\ with\ cahnging\ Phil\ to\ DHCP.png}
    \caption{Resolution: Changing Phil's configuration to DHCP mode for proper network connectivity}
    \label{fig:dhcp_fix}
\end{figure}

\section{Unencrypted FTP Credentials Visible}

The following screenshots demonstrate the vulnerability of unencrypted FTP traffic:

\begin{figure}[H]
    \centering
    \includegraphics[width=0.85\textwidth]{../screenshots/Step\ 2.png}
    \caption{Unencrypted FTP traffic - Username visible in plain text through packet sniffer}
    \label{fig:unenc_user}
\end{figure}

\begin{figure}[H]
    \centering
    \includegraphics[width=0.85\textwidth]{../screenshots/Step\ 3\ password.png}
    \caption{Critical vulnerability: FTP password transmitted in cleartext - visible to anyone on the network}
    \label{fig:unenc_pass}
\end{figure}

\begin{figure}[H]
    \centering
    \includegraphics[width=0.85\textwidth]{../screenshots/Step3\ user.png}
    \caption{Complete credential compromise - Username ``cisco'' exposed in network traffic}
    \label{fig:unenc_full}
\end{figure}

\section{Security Analysis of Part 1}

\textbf{Key Vulnerabilities Observed:}

\begin{itemize}
    \item \textbf{Credential Exposure:} Both username and password are visible in plain text
    \item \textbf{No Encryption:} FTP protocol transmits data without any encryption
    \item \textbf{Passive Eavesdropping:} A cyber criminal can intercept credentials by simply monitoring network traffic
    \item \textbf{No Authentication:} No verification that the connection is legitimate
    \item \textbf{Data Integrity Risk:} Files and messages can be modified in transit
\end{itemize}

\textit{Impact:} An attacker can easily capture login credentials and gain unauthorized access to the FTP server and its contents.

\chapter{Part 2: Configuring VPN Client}

\section{Objective}

Implement VPN protection on Phil's computer to encrypt all traffic destined for the private server.

\section{Initial Ping Test}

Before configuring VPN, we verify network connectivity:

\begin{figure}[H]
    \centering
    \includegraphics[width=0.85\textwidth]{../screenshots/part\ 2\ step1\ ping.png}
    \caption{Initial connectivity test: Ping successful to private server before VPN configuration}
    \label{fig:ping_before_vpn}
\end{figure}

\section{VPN Client Configuration}

Phil's computer is configured with VPN settings to encrypt traffic:

\begin{figure}[H]
    \centering
    \includegraphics[width=0.85\textwidth]{../screenshots/part2\ step\ 2\ VPN\ config.png}
    \caption{VPN Configuration Interface: Setting up IPSec parameters and encryption algorithm}
    \label{fig:vpn_config}
\end{figure}

\section{VPN Client IP Assignment}

Once VPN is configured, Phil's computer receives a virtual IP address from the VPN tunnel:

\begin{figure}[H]
    \centering
    \includegraphics[width=0.85\textwidth]{../screenshots/part2\ step2\ VPN\ config\ client\ ip.png}
    \caption{VPN Client IP Configuration: Virtual IP assigned through secure tunnel}
    \label{fig:vpn_client_ip}
\end{figure}

\section{VPN Connection Established}

The VPN connection is now active and secured:

\begin{figure}[H]
    \centering
    \includegraphics[width=0.85\textwidth]{../screenshots/part2\ step2\ VPN\ popup\ window\ connected.png}
    \caption{VPN Successfully Connected: Secure tunnel established with encryption active}
    \label{fig:vpn_connected}
\end{figure}

\section{VPN Configuration Details}

\textbf{VPN Security Parameters Implemented:}

\begin{itemize}
    \item \textbf{Encryption Protocol:} IPSec (IP Security) with AES (Advanced Encryption Standard)
    \item \textbf{Authentication:} Pre-shared keys and digital certificates
    \item \textbf{Integrity Check:} HMAC (Hash-based Message Authentication Code)
    \item \textbf{Key Exchange:} IKEv2 (Internet Key Exchange version 2)
    \item \textbf{Tunnel Mode:} Encapsulation of entire packets within encrypted wrapper
\end{itemize}

\chapter{Part 3: Verifying Encrypted FTP Traffic}

\section{Objective}

Confirm that FTP traffic is now encrypted when transmitted through the VPN tunnel, making credentials unreadable to potential eavesdroppers.

\section{Network Configuration Check}

Verify that Phil's computer has proper network connectivity through the VPN:

\begin{figure}[H]
    \centering
    \includegraphics[width=0.85\textwidth]{../screenshots/part3\ step\ 1\ ipconfig\ ping.png}
    \caption{Network Status: ipconfig shows active VPN connection with assigned IP}
    \label{fig:part3_ipconfig}
\end{figure}

\section{FTP Connection Through VPN}

Phil sends FTP traffic to the private server. This traffic is now encrypted within the VPN tunnel:

\begin{figure}[H]
    \centering
    \includegraphics[width=0.85\textwidth]{../screenshots/part3\ Step\ 2\ Send\ FTP\ to\ Private\ Server.png}
    \caption{FTP Request Sent: Traffic travels through secure VPN tunnel}
    \label{fig:ftp_through_vpn}
\end{figure}

\section{Encrypted Traffic Captured by Cyber Criminal Sniffer}

The attacker's packet sniffer captures the traffic, but it is now completely encrypted and unreadable:

\begin{figure}[H]
    \centering
    \includegraphics[width=0.85\textwidth]{../screenshots/part3\ step3\ cypted\ message\ for\ cyber\ scriminal\ sniffer\ .png}
    \caption{Captured Encrypted Packet 1: Sniffer shows encrypted payload - credentials are protected}
    \label{fig:encrypted_packet1}
\end{figure}

\begin{figure}[H]
    \centering
    \includegraphics[width=0.85\textwidth]{../screenshots/part3\ step3\ cypted\ message\ for\ cyber\ scriminal\ sniffer\ 2\ .png.png}
    \caption{Captured Encrypted Packet 2: Additional packets remain encrypted - no username/password visible}
    \label{fig:encrypted_packet2}
\end{figure}

\section{Comparison: Security Analysis}

\begin{center}
\begin{tabular}{|l|c|c|}
\hline
\textbf{Security Aspect} & \textbf{Without VPN} & \textbf{With VPN} \\
\hline
Username Visibility & \textcolor{red}{VISIBLE} & \textcolor{green}{ENCRYPTED} \\
Password Visibility & \textcolor{red}{VISIBLE} & \textcolor{green}{ENCRYPTED} \\
Credentials Compromise & \textcolor{red}{HIGH RISK} & \textcolor{green}{PROTECTED} \\
Packet Inspection & \textcolor{red}{EASY} & \textcolor{green}{IMPOSSIBLE} \\
Man-in-the-Middle Attack & \textcolor{red}{POSSIBLE} & \textcolor{green}{PREVENTED} \\
Data Integrity & \textcolor{red}{NO GUARANTEE} & \textcolor{green}{VERIFIED} \\
Authentication & \textcolor{red}{NOT VERIFIED} & \textcolor{green}{VERIFIED} \\
\hline
\end{tabular}
\end{center}

\chapter{Activity Completion}

\section{Final Status}

\begin{figure}[H]
    \centering
    \includegraphics[width=0.85\textwidth]{../screenshots/final\ screenshot\ showing\ everything\ and\ completion\ 40\ of\ 40.png}
    \caption{Activity Complete: All 40 steps successfully completed with 100\% score}
    \label{fig:completion}
\end{figure}

\section{Key Findings}

\textbf{VPN encryption successfully protected sensitive FTP credentials:}

\begin{enumerate}
    \item \textbf{Before VPN:} Attackers could capture username ``cisco'' and password in plain text
    \item \textbf{After VPN:} Same traffic becomes completely unreadable, providing confidentiality
    \item \textbf{Encryption Impact:} All seven layers of network communication are encrypted
    \item \textbf{Authentication:} VPN verified both client and server identity
    \item \textbf{Integrity:} HMAC ensures packets were not tampered with during transit
\end{enumerate}

\chapter{Technical Concepts}

\section{IPSec VPN (IP Security)}

IPSec is a suite of protocols that work together to provide secure communication over IP networks:

\subsection{ESP (Encapsulating Security Payload)}
\begin{itemize}
    \item Encrypts the actual data payload
    \item Provides confidentiality and integrity
    \item Uses algorithms like AES-256 for encryption
\end{itemize}

\subsection{AH (Authentication Header)}
\begin{itemize}
    \item Verifies authenticity and integrity of packets
    \item Detects unauthorized modification
    \item Uses HMAC-SHA-256 or similar algorithms
\end{itemize}

\subsection{IKE (Internet Key Exchange)}
\begin{itemize}
    \item Automatically negotiates security parameters
    \item Establishes shared encryption keys
    \item Authenticates both endpoints
\end{itemize}

\section{Tunnel Mode vs Transport Mode}

\textbf{Tunnel Mode (Used in This Lab):}
\begin{itemize}
    \item Entire original packet is encrypted and encapsulated
    \item New IP header added for tunnel
    \item Complete protection: headers + payload
    \item Typical for VPN gateways and site-to-site connections
\end{itemize}

\textbf{Transport Mode:}
\begin{itemize}
    \item Only payload is encrypted, original IP header remains
    \item Lower overhead
    \item Typical for host-to-host connections
\end{itemize}

\section{Encryption Standards Used}

\subsection{AES (Advanced Encryption Standard)}
\begin{itemize}
    \item Symmetric encryption cipher
    \item 128, 192, or 256-bit key sizes
    \item NIST standard for U.S. government
    \item Mathematically proven secure against known attacks
\end{itemize}

\subsection{HMAC (Hash-based Message Authentication Code)}
\begin{itemize}
    \item Ensures data integrity
    \item Detects tampering in transit
    \item Combined with hash functions (SHA-256)
\end{itemize}

\chapter{Conclusion}

This lab clearly demonstrates the critical importance of VPN encryption in protecting sensitive data:

\begin{itemize}
    \item \textbf{Vulnerability Exposed:} Unencrypted FTP easily compromises credentials
    \item \textbf{VPN as Solution:} Encryption renders eavesdropping impossible
    \item \textbf{Zero Trust Principle:} Never assume network is secure - always encrypt
    \item \textbf{Real-World Application:} Organizations use VPN for remote work, branch connectivity, and cloud access
    \item \textbf{Defense in Depth:} Combine VPN with strong authentication and firewalls
\end{itemize}

\section{Recommendations}

\begin{enumerate}
    \item Always use VPN when connecting to corporate networks remotely
    \item Implement VPN for all sensitive data transmission (financial, medical, personal)
    \item Use strong encryption algorithms (AES-256, not DES or 3DES)
    \item Regularly update VPN firmware and patch security vulnerabilities
    \item Monitor VPN logs for suspicious connection attempts
    \item Implement multi-factor authentication for VPN access
    \item Use site-to-site VPNs for branch office connectivity
\end{enumerate}

\appendix

\chapter{Packet Tracer Activity Details}

\section{Network Components}

\begin{itemize}
    \item Phil's Computer: End client device
    \item Router-FTP-Server: VPN gateway protecting private network
    \item Private FTP Server: Target service behind VPN protection
    \item Attacker's Computer: Packet sniffer for traffic analysis
    \item Cabling: Ethernet connections between devices
\end{itemize}

\section{Steps Summary}

The activity consisted of 40 steps, divided into three main phases:

\begin{enumerate}
    \item \textbf{Part 1 (Steps 1-13):} Configure network and capture unencrypted FTP traffic
    \item \textbf{Part 2 (Steps 14-26):} Configure VPN client with IPSec parameters
    \item \textbf{Part 3 (Steps 27-40):} Verify encrypted FTP traffic through VPN tunnel
\end{enumerate}

\end{document}